\section{Introduction}

\subsection{Human-wildlife interactions}

Human-wildlife conflict (HWC) is one of the great challenges faced by the global conservation movement \cite{Dickman2010b,Redpath2013}. It imparts high social and economic costs for communities living alongside problematic wildlife, and is costly for wildlife as management (legal or illegal) often entails blocking their access to important resources or lethal control \cite{Dickman2010b,Woodroffe2005}. HWC is commonly defined as 'situations occurring when an action by either humans or wildlife has an adverse effect on the other' \cite{Conover2010a}. Using the term human-wildlife conflict to describe such an array of situations can be misleading as arguably animals are unable to consciously enter conflict with humans \cite{Peterson2010b,Redpath2014b}, and so most conflict is "human-human" conflict, between those who prioritise species conservation against those whose interests wildlife threaten \cite{Peterson2010b,Redpath2014b}. To account for this aspect of HWC \citet{Kansky2016} includes a second component to their HWC definition as "conflicts between humans themselves over how to manage the impacts between humans and wildlife".\\

HWC is a global problem occurring in low, middle, and high income countries \cite{Manfredo2004} with huge numbers of examples from every continent with major human habitation \cite{Dickman2014d,Musiani2005,Palmeira2008,Bagchi2006,Prowse2014,Thirgood2016}. HWC comes in innumerable different forms such as: crop raiding; infrastructure damage; competition for prey species, land, and water; timber damage; livestock, and fishery predation; disease transmission; and human-injuries, and deaths \cite{Arlet2007b,Messmer2000,Conover1997a,woodroffe2005people,Thirgood2005}. HWC involves wildlife from a range of taxonomic groups (e.g. terrestrial and marine mammals \cite{Loe2004a,Butler2015}, birds \cite{Redpath1997}, reptiles \cite{Chaves2015}, fish \cite{Freitas2016} and insects \cite{Cease2015}) and while rural human societies are the historically most affected, HWC is an increasing problem in urban societies \cite{Messmer2000}. The range of human groups affected is immense; from subsistence pastoralists \cite{Dickman2014d} and agriculturalists \cite{Arlet2007b} in low income countries to metropolitan residents \cite{Conover1997a} and landed nobility in high income countries \cite{Redpath1997}.\\

Competition and conflict between humans and wildlife has existed throughout human history and driven many species to local and global extinction \cite{Graham2005,Woodroffe2005a}. Historically HWC was resolved by the legal (often state sponsored) or illegal persecution of problem species or mass habitat clearance \cite{Woodroffe2005a,treves2005evaluating}. These actions have had huge impacts on biodiversity through the persecution of keystone species, i.e. elephants, prairie-dog and apex predators, such as wolves, \cite{Woodroffe2005a,Kotliar1999,sinclair1995equilibria,Ripple2001}. Wildlife management has traditionally been considered a "rural or agricultural problem" \cite{Messmer2000} with wildlife managers mainly responsible for destroying wildlife that threatened human interests \cite{treves2005evaluating}. However, as the wider economic, social and cultural benefits of wildlife are increasingly recognised (particularly by urbanites) the pressure on managers to protect wildlife has increased \cite{treves2005evaluating}. The direct human-wildlife interfaces of HWC systems mostly involve local communities and the species they are in competition with \cite{Thirgood2005}. Increasingly these communities are trapped between damaging wildlife and a myriad of local, national and international institutions that attempt to dictate or control their relationship and interactions with their local environment \cite{naughton2005socio}.\\

The majority of past work has used a 'human-wildlife' conflict paradigm \cite{Dickman2010b,Peterson2010b,Redpath2014b} which focuses on technological solutions that aim to stop conflict events from happening \cite{naughton2005socio} and misses many of the social and ecological complexities that drive HWC systems \cite{Redpath2014b}. Re-framing human-wildlife conflict as human-human conflict re-focuses mitigation strategies away from technological approaches to stopping wildlife 'attacking' humans to one which tries to understand and reconcile the different human attitudes towards wildlife \cite{Redpath2014b}.\\

\subsection{Attitudes and tolerance towards wildlife in conflict systems}

Conservation biologists often make three assumptions about conflict systems: (A) the level of wildlife damage is directly related to the level of conflict engendered; (B) the level of conflict elicits a proportionate response; and (C) that altering the response to conflict will have proportionate conservation effects  \cite{Dickman2010b}. This however ignores many of the attitudinal factors that also come into play and influence peoples perceptions of wildlife and their conflict with them \cite{Dickman2010b}. Recent studies show that with a number of species the damage they cause is not directly proportional to people's attitudes or tolerance towards them or the responses they elicit \cite{Dickman2010b,Kansky2014d,Kansky2014e,Kansky2016}, carnivores in particular elicit disproportionately negative attitudes \cite{Kansky2014e} and in some systems species are discriminated against despite having an overall positive impact on peoples livelihoods \cite{Dickman2010b,Prowse2014}.\\

There is a wide literature on attitudes and tolerance towards problem species \cite{Kansky2014d,Kansky2014e} however much of this literature has been guided by conservationist's intuition and not psychological theory, thus there is much confusion and cross-over between the similar but different concepts of tolerance, acceptance and attitudes towards wildlife \cite{Bruskotter2015}. \citet{Bruskotter2012} defines tolerance as "passive acceptance of a wildlife population" and intolerance is "when an animal or population becomes unacceptable". \citet{Bruskotter2015,Treves2012b} further define intolerance as made up of two aspects: "prejudicial" attitudes and "discriminatory" behaviour. While \citet{Ajzen1991}'s theory of planned behaviour highlights that prejudicial attitudes do not always lead to discriminatory behaviour, \citet{Bruskotter2015} found strong correlation between attitudinal and behavioural measures of intolerance. Thus in some cases the use of attitudinal measures to assess tolerance may be the most suitable method in areas where behavioural measures are likely to be misreported or overly sensitive \cite{Bruskotter2015}.\\

There are a number of different frameworks that outline different psychological, social and economic factors that contribute to tolerance towards species (figure \ref{fig:A_toleranceframe}). \citet{Kansky2016}'s framework outlines inner and outer models that combine to influence tolerance. The outer model is made up of the experiences a person has had with a species and the benefits and costs of living alongside the species. The inner model consists of 11 variables which influence how a person perceives the costs and benefits. \citet{Bruskotter2014d}'s and \citet{Inskip2016}'s frameworks share similar variables (perceptions of risk; experience of the species; and costs, and benefits, which \citet{Inskip2016} include in "beliefs") however \citet{Bruskotter2014d} include people's feeling of control over the risks that arise from the species and their trust in the ability of those charged with managing the species. This is supported by other literature where people's relationship with wildlife authorities is an important variable in how they perceive those species and levels of conflict \cite{Dickman2010b}.\\

If we are to reconcile different human attitudes towards wildlife it is important to understand the drivers behind different attitudes and levels of tolerance towards 'problematic' species, something that has been relatively under explored in the HWC literature \cite{Dickman2010b}.\\

\subsection{Human carnivore conflict}

Large carnivores are an ecologically, economically and socially important group of animals \cite{Wolf2016,Ripple2014a}. Through a number of top down ecological mechanisms (for example mesopredator release \cite{brashares2010ecological,Crooks1999a,Ripple2013}, direct predation and landscapes of fear \cite{Ripple2004,Schmitz1997}) carnivores can have wide ranging impacts on the ecological systems they exist in. Through these ecological processes carnivores can also provide economically valuable ecosystem services (for example controlling populations of pest animals \cite{brashares2010ecological,Prowse2014,Packer2005}). In many areas large carnivores also offer direct economic benefits as they are often the most highly sought after animals, and thus a primary driver, for photographic tourism \cite{Lindsey2007a,Maciejewski2014}. The importance of this economic contribution can be magnified as wildlife based tourism is often prominent in less economically developed areas that lack other economic opportunities \cite{Ashley2000}. Large carnivores are also socially important as they play important cultural roles in a number of different societies \cite{Hazzah2009a,Shen1982,coggins2003tiger,Kellert1996} and many people derive pleasure purely from the knowledge that large carnivores exist, pleasure that exists independently from viewing or "consuming" the species \cite{Stevens2016}.\\

Despite their importance large carnivore populations are under intense pressure around the world and many are threatened with extinction \cite{ray2013large,Wolf2016,Ripple2014a}. Of the 31 largest species of Carnivora (excluding pinnipeds) 24 are decreasing in number and 19 are classed as vulnerable, endangered or critically endangered \cite{Ripple2014a}. Large carnivore's place at the top of the food chain means they require large home ranges containing large bodied prey, exist at low densities, and have low reproduction rates making them naturally rare species \cite{Ripple2014a}. These factors mean they also come into competition with humans for space and prey and are thus especially at risk from human activities such as habitat destruction, human induced prey depletion and persecution \cite{Woodroffe1998}.\\

The high existence values given to large carnivores by an international audience are rarely shared at the local level by communities who often pay high social and economic costs for living alongside carnivores \cite{Dickman2011,Abade2014h}. Predation on livestock by carnivores can impose severe economic costs on those they live nearby \cite{loveridge2010people}. In parts of Botswana livestock owners lose an average 5.5\% of their livestock to predators (with some respondents losing 100\% of their stock) \cite{Hemson2009b}, in northern Tanzania and Bhutan villagers report losing on average over two thirds of their annual cash income to carnivores \cite{Holmern2007a,Wang2006a} while on Brazilian ranches predators were responsible for 19\% of cattle mortality, making up loses of 4.2\% of the ranches commercialised meat \cite{Palmeira2008}.\\

Those living alongside carnivores often live in poor, remote areas and lack economic opportunities outside of livestock keeping. A heavy reliance on livestock and weak economic position makes people particularly vulnerable to the economic shock of livestock loses \cite{Bagchi2006}. The unpredictable and potentially catastrophic nature of carnivore attacks (for instance a household suffering a "surplus killing" where a carnivore kills many stock in one attack) make it hard for households to protect themselves economically from these blows and can cause intense hostility towards carnivores \cite{Dickman2011,Musiani2005}. In some areas social capital may help protect people from these heavy shocks however in many areas where carnivores occur there are high levels of wealth inequality and those living in the areas most at risk of carnivore attacks are often the economically and socially poorest and weakest. The poorest are least able to cope with economic shocks, and stochastic losses of livestock can be instrumental in pushing people into poverty traps and keeping them there \cite{Lybbert2015}.\\

Carnivores can also impose indirect costs on those they live alongside. Opportunity costs for time spent defending livestock from carnivores can be very high \cite{woodroffe2005people}. Conflict can further push poor families and their children into poverty traps if livestock loses makes them unable to afford school fees or if children forgo schooling in order to protect livestock \cite{Dickman2011}. Human fatalities are another cost imposed by a range of large carnivores across the globe (table \ref{table:Deaths}) \cite{Loe2004a}. People killed in human-wildlife conflict events are often from weaker socio-economic sectors of society \cite{Das2011}. The geographic distribution of attacks on humans highlights discrepancies between the experiences of urban societies (who place high existence values on carnivores) and rural societies in their dealings with large carnivores. In urban societies most attacks occur when people are engaged in recreational outdoor activities and thus choose to place themselves in risky situations whereas in rural societies most attacks occur during everyday domestic activities \cite{Loe2004a}.\\

\begin{table}[h]
	\small
	\begin{center}
		\begin{tabular}{l l}
			\hline \hline		
			Species 				& Human deaths\\ \hline
			Tiger (\textit{Panthera tigris})				& 12,599\\
			Leopard 	(\textit{Panthera pardus})			& 840\\
			Wolf	 (\textit{Canis lupus})			 		& 607\\
			Lion	 (\textit{Panthera leo})					& 552\\
			Brown bear (\textit{Ursus arctos}) 			& 313\\
			\hline \hline						
		\end{tabular}
		\caption{The five large carnivore species responsible for the most human deaths in the 20th century \cite{Loe2004a}.}
	\label{table:Deaths}
	\end{center}
\end{table}

Livestock loses to carnivores can be even more severe than just the upfront economic loss as livestock often hold intangible value far beyond that of their direct economic value \cite{Kansky2014e}. In many rural communities where there is little or no access to formal credit and insurance institutions livestock provide an investment and safety net that are used to fulfil this role \cite{Kurosaki1995,Andrew2003}. Livestock can also have high social and cultural values, for instance the Maasai of East Africa value their cattle for social, political, religious and cultural reasons and much of the Maasai's cultural identity is defined through their relationship with livestock \cite{Galaty2016a}.\\

Carnivores are widely persecuted as a result of the costs they impose on human communities \cite{Dickman2010b,dickman2013human,loveridge2010people,Woodroffe2005} and they often elicit disproportionately harsher responses when compared to the damage they cause and attitudes towards other species \cite{Kansky2014e,Dickman2010b}. They are often killed opportunistically or pre-emptively to reduce carnivore populations or in direct retaliation for a specific attack on livestock or people \cite{Thirgood2005}. Human persecution is one of the greatest threats faced by large carnivores and even within protected areas humans are usually the single biggest cause of adult mortality \cite{Woodroffe1998}. Reducing levels of conflict and mitigating carnivore persecution has been highlighted as one of the most pressing concerns for large carnivore conservation globally \cite{Woodroffe1998,Ray2005}.\\

\subsection{Ecosystem impacts of human\-carnivore conflict}
Human-wildlife conflict often takes place in complex and poorly understood ecological systems. This has led to species being persecuted despite having an overall positive impact on peoples livelihoods and unintended negative ecological and economical consequences from the control or extermination of certain species \cite{Dickman2010b,Prowse2014}.\\

\citet{Berger2001} recorded the wide ranging ecological impacts that occured following the eradication of wolves and grizzly bears from Grand Teton National Park. With no predation pressures moose numbers increased which led to significantly changed riparian vegetation structures and reduced numbers of avian neotropical migrants. The wide reaching implications of wolf eradication and re-introduction is well documented in Yellowstone National Park \cite{Ripple2012}. Persecution of Prairie dogs throughout the 20th century due to perceived competition with cattle reduced populations by around 98\% \cite{Kotliar1999,Whicker1988}, this resulted in the decimation of Black Footed Ferret populations \cite{Kotliar1999} and has been linked to reduced nutritional quality of rangeland vegetation for large herbivores and livestock \cite{Whicker1988}. \citet{Prowse2014} \& \citet{Allen2015a} investigated the ecological and economic consequences of Dingo control in Australian rangelands and found under certain cattle stocking densities dingo's increased forage availability to cattle by controlling kangaroo numbers, the economic gains from increased forage outweighed the costs of calves lost to dingo predation.\\

While the human-carnivore conflict literature focuses on the costs of living alongside carnivores there is little in the conflict literature on the ecosystem services that carnivores offer \cite{Ripple2014a}. There are few studies that link the very rich literature on top-down impacts carnivores have on ecosystems and the cascades that can result in their eradication \cite{Ripple2014,Crooks1999a,Ripple2014a} with human-carnivore conflict. A number of studies investigate the impacts of carnivores on carbon sequestration \cite{Wilmers2012,Schmitz2014}, water quality \cite{Beschta2012a}, and nutrient cycling \cite{Wilmers2003}. However, there are relatively few studies that investigate ecosystem services that have a more direct impact on local communities, such as disease prevalence in livestock \cite{Packer2003}, rangeland quality \cite{Prowse2014,Allen2015a} or herbivore crop raiding \cite{brashares2010ecological}. Improving our understanding of the ecological aspects of HWC systems will allow us to better understand the true costs and benefits of living alongside carnivores and how these are divided amongst different sectors of society.

\subsection{Spatial and temporal dynamics of human-carnivore conflict systems}

Reducing the numbers of carnivore attacks on livestock is a priority for two reasons. Firstly to reduce the economic costs of living alongside carnivores amongst communities who can often ill afford the economic and social shocks associated with carnivore attacks. Secondly as even though livestock loses are only one factor of many that influence people's attitudes towards carnivores \cite{Dickman2010b,Inskip2016,Bruskotter2014d} livestock depredations can trigger negative attitudes and responses towards carnivores that persist for a long time \cite{Marker2003,Dickman2014d}.\\

A wide range of ecological, social, spatial and temporal factors influence the risk of livestock being attacked by carnivores \cite{Miller2015}. Predator-prey dynamics can play an important role in depredation risk, with density of both livestock and wild prey important determinants of risk \cite{Hemson2003,Zarco-Gonzalez2013}, however different studies have found wild prey density to be positively \cite{Kolowski2006,Treves2015,Zarco-Gonzalez2013} and negatively \cite{Hemson2003} associated with depredation risk. The age and type of livestock can also influence the overall risk of attack; the carnivore species livestock are at risk from; and the time, day and place where attacks might occur \cite{DeAzevedo2007,Ogada2003}. \\

A variety of landscape features have been linked to changing attack risks such as: distance to forests \cite{DeAzevedo2007}; proportion of crop lands, coniferous forest, herbaceous wetlands, and open water \cite{Treves2015}; over-all vegetative cover, and altitude \cite{Zarco-Gonzalez2013}; and distance to rivers, elevation, and percentage tree-cover \cite{Abade2014h}. Climatic conditions can also play a role, season and localised rainfall interact with carnivore species and livestock husbandry to influence predation risk \cite{Kissui2008,Abade2014h}. A number of human factors are also important with roads, protected areas, farm sizes, population densities and the structure of human settlements all influential factors \cite{Treves2015,Zarco-Gonzalez2012,Holmern2007a}.\\

Livestock husbandry methods have been found by some studies to be highly influential in mediating risk of attack. \citet{Woodroffe2007} found presence of men and dogs as well as the design of livestock enclosures reduced attack risk while presence of scarecrows increased risk of attack. \citet{Ogada2003} found enclosure construction and the presence of watchdogs and human activity all reduced losses to predators. Conversely, while \citet{Kolowski2006} found different types of enclosures to influence risk of attack from different carnivores, non-traditional fences, dogs and human activity did not influence the overall risk of attack. \citet{Abade2014h} also found husbandry techniques did not influence predation risk.\\

Factors that influence attack risk also differ by carnivore species, risk of attack from different carnivore species in the same landscapes can be affected by time of day and season \cite{Kissui2008,}; distance to protected areas \cite{Holmern2007a}, livestock husbandry \cite{Woodroffe2007,Kolowski2006} and the structure of human settlements \cite{Kolowski2006}. While most studies look at the risk of livestock being attacked \citet{Packer2005} investigating the risk of human predation found attacks correlated with low overall wild prey density but high bush pig density. Attacks were more likely to occur in the harvest season and most victims were men.\\

Improving our understanding of the specific factors that underlie attack risk for different species (carnivore and livestock) in different environments and human societal structures will be fundamental in helping pastoralists to protect their livestock from attacks.\\

\subsection{Carnivores and conflict in Tanzania}

The Tanzanian guild of large carnivores contains six species all with declining populations across the continent (table \ref{table:EACarn}) \cite{Winterbach2013}. Despite the charismatic nature of these species and the importance of the Tanzanian populations (for instance Tanzania holds over 40\% of the world's remaining lions \cite{Riggio2013}) apart from a handful of well studied populations very little is known about their distribution and population trends across the country \cite{TAWIRI2009}. Conflict with humans and subsequent persecution is listed as a major, if not the greatest, threat to these species both across the African continent and within Tanzania \cite{IUCN2016,Ray2005,TAWIRI2009}. Across the country these carnivores impose severe costs to communities they come into conflict with \cite{Packer2005,Dickman2008,Kissui2008}, however, they are economically important as a major attractant of international tourists who provide almost a quarter of the countries foreign exchange \cite{Bank2015}. 
\\

%African wild dog (\textit{Lycaon pictus}) are classed as endangered and are estimated to exist in 6\% of their historic range with population declines of 17\% between 1997 and 2012 \cite{Woodroffe2012a,iucn2007regional}. Lions (\textit{Panthera leo}) and cheetah (\textit{Acinonyx jubatus}) are classed as vulnerable; lions are estimated to have declined by 43\% over the last two decades, and currently exist in 8\% of their historic range \cite{Bauer2016} while Cheetahs are estimated to exist in only 10\% of their historic range in Africa. Leopards (\textit{Panthera pardus}) and striped hyaenas (\textit{Hyaena hyaena}) are near threatened; leopards are estimated to have lost 48-67\% of their historic range in Africa \cite{jacobson2016leopard} and striped hyaena populations are are expected to decline by almost 10\% over the next 3 generations \cite{AbiSaid2015}. Spotted hyaena (\textit{Crocuta crocuta}) are classified as least concern \cite{IUCN2016} although their global population is declining \cite{Bohm2015}.\\

\begin{table}[h]
	\small
	\begin{center}
		\begin{tabular}{p{2.1cm} p{2.7cm} p{3cm} p{3cm} p{3cm}}
			\hline \hline		
			Species 			& IUCN Status			& Population decline 	& \% of historical range & Level of protection of current populations\\ \hline
			Lion	 (\textit{Panthera leo})				& Vulnerable 		& 43\% in two decades 		& 8\% 		& - \\
			Leopard 	(\textit{Panthera pardus})		& Near threatened 	& -	 						& 33-52\% 	&17\% of current range protected\\
			Cheetah (\textit{Acinonyx jubatus})		& Vulnerable			& - 							& 10\% 		& 76\% of current range protected\\
			Wild Dog (\textit{Lycaon pictus})		& Endangered			& 17\% between 1997 and 2012	& 6\%		& - \\
			Spotted Hyaena (\textit{Crocuta crocuta})& Least Concern		& - 							& 73	\%		& - \\
			Striped Hyaena (\textit{Hyaena hyaena})	& Near threatened	& 10\% decline expected over next 3 generations & 62\% & - \\
			\hline \hline						
		\end{tabular}
		\caption{Where available the global population and range declines, IUCN status and level of protection current populations receive for the six large East African carnivores \cite{Loe2004a,Woodroffe2012a,iucn2007regional,Bauer2016,jacobson2016leopard,AbiSaid2015,Bohm2015,IUCN2016,Ray2005,Durant2015}.}
	\label{table:EACarn}
	\end{center}
\end{table}

\subsection{The Ruaha landscape}
 
Tanzania's Ruaha landscape is an internationally important site for African carnivores, it contains 10\% of the world's lions, one of four East African cheetah populations larger than 200 individuals, the world's third largest population of African wild dogs, and globally important populations of leopards and spotted Hyaenas \cite{Dickman2014d}. The landscape contains Ruaha National Park but also a mix of game reserves, wildlife management areas and village land encompassing an area of approximately 50,000km$^2$. As well as these important carnivore populations, around 40,000 people live in village land that sits on the south west border of Ruaha National Park and within the Ruaha landscape. These communities are culturally complex containing at least 35 different ethnic groups many of whom rely predominantly or solely on livestock for their livelihoods  \cite{Abade2014h}. Through the killing of livestock, and sometimes people, Ruaha's carnivore populations impart economic and social costs on these local communities who respond through retaliatory and pre-emptive killings \cite{Dickman2010b}.\\

Lions, cheetahs, leopards, African wild dog and spotted hyenas are all cited by villagers living around the park as being problem animals \cite{Dickman2008}. Particularly high levels of conflict with these animals means most villagers around the park want populations of these carnivores to decline or become locally extinct \cite{Dickman2008}. While livestock predation was cited as the main reason behind peoples dislike for carnivores \citet{Dickman2008} found that there were numerous other factors that also influenced peoples attitudes and perceptions of conflict such as their cultural, economic and religious backgrounds. The importance of Ruaha's carnivore populations and the complexity and intensity of the conflict between those living in the villages surrounding the park and the carnivores highlight the importance of continued research into human-carnivore conflict in the Ruaha landscape.\\

\subsection{Ruaha Carnivore Project}

The Ruaha Carnivore Project (RCP) is a conservation organisation that works in and around Ruaha National Park to improve our understanding of carnivore ecology in the Ruaha landscape and reduce levels of human-carnivore conflict around the park. RCP has been working in the land around Ruaha since 2009, slowly expanding the numbers of villages it works in over that time. The project currently has projects running in 11 villages on the eastern edge of the national park. During the 7 years since RCP started it has built a a number of large datasets on levels of conflict and attitudes towards carnivores that are relevant to this project. RCP has also been successful in building working relationships amongst the pastoralist and agriculturalist communities that live in this area. \\